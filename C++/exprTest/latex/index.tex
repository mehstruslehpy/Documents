{\bfseries W\+H\+AT T\+H\+IS IS\+:} 
\begin{DoxyItemize}
\item Exprtest is a program based off of a design from the design patterns book (the gang of four one). 
\item Currently this program has two modes of operaton\+: 
\begin{DoxyItemize}
\item Calculator mode\+: calculates boolean formulas using the supported operators 
\item Proof mode\+: proves a propositional argument or provides a counter argument 
\end{DoxyItemize}
\item This program was written and tested on linux I cannot vouch for whether or not it will run on windows 
\item Input can be done in three ways\+: 
\begin{DoxyItemize}
\item via the here document generator tool in the proof\+Translator and calc\+Translator directories 
\begin{DoxyItemize}
\item type out the well formed formulas and specify when you are entering the conlusion via the prompts  
\end{DoxyItemize}
\item via the built in repl in the \char`\"{}main\char`\"{} binary 
\begin{DoxyItemize}
\item Start with specifying a unary or binary proposition 
\begin{DoxyItemize}
\item For unary\+: 
\begin{DoxyItemize}
\item it can be a variable or the negation of a formula 
\end{DoxyItemize}
\end{DoxyItemize}
\begin{DoxyItemize}
\item For binary\+: 
\begin{DoxyItemize}
\item Pick between conditional, and, or or formulas 
\item arguments get filled in from left to right 
\end{DoxyItemize}
\end{DoxyItemize}
\end{DoxyItemize}
\item via automating the repl by hand using a here document 
\begin{DoxyItemize}
\item type out the correct sequence of arguments that you would type into the repl 
\item I did not invent this method of input but I found it very useful during testing 
\item all 60 test arguments use this method see the first example in the valid arguments directory 
\end{DoxyItemize}
\end{DoxyItemize}
\end{DoxyItemize}

{\bfseries Well Formed Formulas\+:} 
\begin{DoxyItemize}
\item All formulas except negations and single variables are parenthesized 
\item If you use the here doc generator you must use single letter upper case variables 
\begin{DoxyItemize}
\item you should do this for the repl or hand written here docs but you don\textquotesingle{}t have to
\end{DoxyItemize}
\item For general propositions @ and \# the supported formulas are 
\begin{DoxyItemize}
\item \textquotesingle{}(\#v@)\textquotesingle{} for or expressions 
\item \textquotesingle{}(\#\&@)\textquotesingle{} for and expressions 
\item \textquotesingle{}(\#-\/$>$@)\textquotesingle{} for conditional expressions 
\item \textquotesingle{}$\sim$@\textquotesingle{} for not expressions 
\item \textquotesingle{}\#\textquotesingle{} for single variables 
\item \textquotesingle{}\#\textquotesingle{} and \textquotesingle{}@\textquotesingle{} can be any of the expression types above 
\item For Example $\sim$((AvB)\&$\sim$$\sim$($\sim$\+C-\/$>$$\sim$D)) is a well formed formula 
\end{DoxyItemize}
\end{DoxyItemize}

{\bfseries The Inference Rules\+:} 
\begin{DoxyItemize}
\item \{(AvB), $\sim$A $\vert$-\/ B\} for v\textquotesingle{}s 
\item \{(A\&B) $\vert$-\/ A, B\} for \&\textquotesingle{}s 
\item \{(A-\/$>$B), A $\vert$-\/ B\} for -\/$>$\textquotesingle{}s (modus ponens) 
\item \{(A-\/$>$B), $\sim$B $\vert$-\/ $\sim$A\} for -\/$>$\textquotesingle{}s (modus tollens) 
\item \{$\sim$$\sim$A $\vert$-\/ A\} for double negation 
\item \{$\sim$(AvB) $\vert$-\/ $\sim$A, $\sim$B\} for negated v\textquotesingle{}s 
\item \{$\sim$(A\&B), A $\vert$-\/ $\sim$B\} for negated \&\textquotesingle{}s 
\item \{$\sim$(A-\/$>$B) $\vert$-\/ A, $\sim$B\} for negated -\/$>$\textquotesingle{}s 
\end{DoxyItemize}

{\bfseries Inference Rules and Proofs\+:} 
\begin{DoxyItemize}
\item Inference rules are listed to the right of the line to which they apply in \textquotesingle{}\{\}\textquotesingle{} brackets. 
\item Each inference rule used in a proof specifies also the lines used to make the inference. 
\item I use the turnstile \textquotesingle{}A $\vert$-\/ B\textquotesingle{} to mean from the general formula A the formula B is inferred 
\item each inference rule listed uses A,B as general formulas, do not confuse these with the specific letters/variables you choose to use in your proof. 
\item Every proof is done by contradiction if possible, periodically the prover checks for a counterargument this way so long as a counterargument or proof exists my prover will find it (given enough time and memory) 
\item the prover can also make assumptions for proofs that require them, this is done periodically similar to the counterargument checks if needed. 
\item this proof method is heavily based off the system used in Harry Genslers intro to logic book which I thought was cool (obviously) 
\end{DoxyItemize}

{\bfseries M\+I\+SC\+:} 
\begin{DoxyItemize}
\item to build the program run the command \char`\"{}make all\char`\"{} 
\item after you build the program I have created a set of test proofs under the invalidproofs and validproofs directories just do ./prooftxtfile to run them if necessary you may need to give the file execute permissions 
\item to use the proof translator after building the program go into the proof\+Translator directory run the program translator and follow the prompts after you are done a file will have been created in that directory for you to run as explaine above (you may need to give this file execute permissions too) 
\end{DoxyItemize}