\documentclass{article}
\usepackage{graphics, fitch, amssymb, ragged2e}
\begin{document}
\begin{center}
\textbf{Signed Values}

\end{center}

\begin{flushleft}
\textbf{Explanation:} In order to better understand the proof given in the notes I have tried reworking it for myself. I wanted to also test fitch.sty a package for writing logic proofs for latex. The end result is similar to a two column proof. The premises/assumptions are the statements at the start, new statements are on the left with their respective justifications and referenced lines on the right.
\newline
\newline
\textbf{What I am trying to prove:}
$$((x_{m-1})?(-(2^m - \sum_{i=0}^{m-1} x_i 2^i)):(\sum_{i=0}^{m-1} x_i 2^i)) \Rightarrow ((\sum_{i=0}^{m-2} x_i 2^i) - x_{m-1}2^{m-1})$$
\newline
\newline
\textbf{How:}
A ternary expression effectively has the form "if a then b follows and if not a then c follows." I have chosen to treat this as a two case conditional proof starting first with $x_{m-1}=1$ as case 1.
\newline
\newline
\textbf{Case 1: $x_{m-1}=1$}
\end{flushleft}
\begin{equation*}
\begin{fitch}
\fb x_{m-1}=1 & Assumption \\
\fj (x_{m-1})?(-(2^m - \sum_{i=0}^{m-1} x_i 2^i)):(\sum_{i=0}^{m-1} x_i 2^i) & Assumption \\
\fa -(2^m - \sum_{i=0}^{m-1} x_i 2^i) & From 1 and 2, defn of ternary stmt \\
\fa (\sum_{i=0}^{m-1} x_i 2^i) - 2^m & From 3, Multiplication\\ 
\fa (\sum_{i=0}^{m-2} x_i 2^i) +x_{m-1}2^{m-1} - 2^m & From 4, Expansion of a Series\\
\fa (\sum_{i=0}^{m-2} x_i 2^i) - (2^m - x_{m-1}2^{m-1}) & 5, Factoring \\
\fa (\sum_{i=0}^{m-2} x_i 2^i) - (2^m - (1)2^{m-1}) & 6 and 1, Multiplicative identity \\
\fa (\sum_{i=0}^{m-2} x_i 2^i) - (2^m -2^{m-1}) & 7, Simplification \\
\fa (\sum_{i=0}^{m-2} x_i 2^i) - (2^m -2^m 2^{-1}) & 8, Rules of Exponents \\
\fa (\sum_{i=0}^{m-2} x_i 2^i) - 2^m(1 - \frac{1}{2}) & 9, Factoring \\
\fa (\sum_{i=0}^{m-2} x_i 2^i) - 2^m(\frac{1}{2}) & 10, Simplification \\
\fa (\sum_{i=0}^{m-2} x_i 2^i) - 2^{m-1} & 11, Rules of Exponents \\
\fa (\sum_{i=0}^{m-2} x_i 2^i) - (1)2^{m-1} & 12, Multiplicative identity \\
\fa (\sum_{i=0}^{m-2} x_i 2^i) - x_{m-1}2^{m-1} & 1 and 13, Substitution \\
\end{fitch}
\end{equation*}
\begin{flushleft}
Which completes the proof for Case 1.
\newline
\pagebreak
\newline
\textbf{Case 2: $x_{m-1}=0$}
\end{flushleft}
\begin{equation*}
\begin{fitch}
\fb x_{m-1}=0 & Assumption \\
\fj (x_{m-1})?(-(2^m - \sum_{i=0}^{m-1} x_i 2^i)):(\sum_{i=0}^{m-1} x_i 2^i) & Assumption\\
\fa \sum_{i=0}^{m-1} x_i 2^i & From 1 and 2, defn of ternary stmt\\
\fa (\sum_{i=0}^{m-2} x_i 2^i)+x_{m-1}2^{m-1} & From 3, Expansion of a Series\\
\fa (\sum_{i=0}^{m-2} x_i 2^i)+(0)2^{m-1} & 1 and 4, Substitution \\
\fa (\sum_{i=0}^{m-2} x_i 2^i)+0 & 5, Multiplying by 0 gives Additive Identity \\
\fa \sum_{i=0}^{m-2} x_i 2^i & 6, Additive Identity \\
\fa (\sum_{i=0}^{m-2} x_i 2^i)-(0)2^{m-1} & 7, Additive Identity \\
\fa (\sum_{i=0}^{m-2} x_i 2^i)-x_{m-1}2^{m-1} & 8 and 1, Substitution\\
\end{fitch}
\end{equation*}
\begin{flushleft}
Which completes the proof for Case 2 and ends the proof.
\end{flushleft}
\end{document}